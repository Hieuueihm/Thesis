\clearpage
\phantomsection

\addcontentsline{toc}{chapter}{Tóm tắt}
\chapter*{\fontsize{13}{13}\selectfont{Tóm tắt}}
\fontsize{12}{12}\selectfont{
\noindent\textbf{Tóm tắt:}
Đồ án thực hiện thiết kế ở mức RTL một bộ trích xuất đặc trưng dựa trên thuật toán MRELBP (Median Robust Extended Local Binary Pattern). Phương pháp MRELBP được lựa chọn do khắc phục được hạn chế của LBP truyền thống trong việc nhận diện cấu trúc vĩ mô và tính nhạy với nhiễu. Việc thiết kế phần cứng chuyên biệt giúp tăng tốc xử lý trích xuất đặc trưng, đáp ứng với các yêu cầu cho hệ thống thời gian thực. Hệ thống SoC được triển khai trên nền tảng phần cứng ZCU106 Ultrascale và đánh giá độ chính xác bằng tập dữ liệu Outex-TC với ảnh kích thước 128x128. Kết quả thực thi cho thấy thời gian trích xuất đặc trưng nhanh gấp khoảng 1700 lần phiên bản xử lý bằng phần mềm với ngôn ngữ C++, độ chính xác với các tinh chỉnh với phiên bản phần cứng không thay đổi nhiều so với các phiên bản phần mềm. 

\vspace{0.5cm}
\noindent\textit{\textbf{Từ khóa:}} \textit{LBP, MRELBP, ZCU106, SoC.}
}